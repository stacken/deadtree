\documentclass[fontsize=11pt,enlargefirstpage,firstfoot=false,a4paper,pagenumber=no]{scrlttr2}

\usepackage[english]{babel}
\usepackage[english]{isodate}
\usepackage{multicol}

% font and input setup
\usepackage[utf8]{inputenc}
\usepackage[T1]{fontenc}

\usepackage{datetime}
\renewcommand{\dateseparator}{-}
\newcommand{\todayiso}{\the\year \dateseparator \twodigit\month \dateseparator \twodigit\day}

\setkomavar{date}{\todayiso}
\setkomavar{signature}{Datorföreningen Stackens styrelse, genom Stefan Berggren}

\let\raggedsignature=\raggedright

\makeatletter
\@setplength{toaddrvpos}{30mm}
\@setplength{toaddrhpos}{130mm}

% The space to sign my name
\@setplength{sigbeforevskip}{10mm}
\makeatother


\setkomavar{subject}{Kallelse till Stackens höstmöte}

\begin{document} 
\begin{letter}{FULLNAMEANDADRESS}

\opening{Härmed kallas du som är medlem i Datorföreningen Stacken till höstmöte, torsdag 19 november 2015 kl. 19.30 i en sal nära Stackenlokalen.}

Var god betala din medlemsavgift om den inte redan är inbetald (eller du är hedersmedlem). Om du betalar den i tid (inkommen innan höstmötet) har du rätt att rösta på mötet. För arbetande är avgiften 215 kronor, för övriga är den 115 kronor.

~\\
Förslag till dagordning:

\begin{multicols}{2}
\begin{enumerate}
	\itemsep0em
	\item  Mötets öppnande
	\item  Val av mötesordförande
	\item  Frågan om mötet är stadgeenligt utlyst
	\item  Val av justeringsmän tillika rösträknare
	\item  Val av mötessekreterare
	\item  Frågan om dagordningens godkännande
	\item  Tillkännagivande av röstlängd
	\item  Val av styrelse
	\item  Fastställande av firmatecknare
	\item  Val av revisorer
	\item  Val av valberedning
	\item  Fastställande av årsavgift
	\item  Fastställande av budget
	\item  Motion om stadgeändring: kallelser
	\item  Övriga frågor
	\item  Mötets avslutande
\end{enumerate}
\end{multicols}

\closing{}
\enlargethispage{3\baselineskip}

\end{letter}
\end{document}
