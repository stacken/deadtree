\documentclass[fontsize=11pt,enlargefirstpage,firstfoot=false,a4paper,pagenumber=no]{scrlttr2}

\usepackage[english]{babel}
\usepackage[english]{isodate}
\usepackage{multicol}

% font and input setup
\usepackage[utf8]{inputenc}
\usepackage[T1]{fontenc}

\usepackage{datetime}
\renewcommand{\dateseparator}{-}
\newcommand{\todayiso}{\the\year \dateseparator \twodigit\month \dateseparator \twodigit\day}

\setkomavar{date}{\todayiso}
\setkomavar{signature}{Datorföreningen Stackens styrelse, genom Stefan Berggren}

\let\raggedsignature=\raggedright

\makeatletter
\@setplength{toaddrvpos}{30mm}
\@setplength{toaddrhpos}{130mm}

% The space to sign my name
\@setplength{sigbeforevskip}{10mm}
\makeatother


\setkomavar{subject}{Kallelse till Stackens vårmöte}

\begin{document} 
\begin{letter}{FULLNAMEANDADRESS}

\opening{Härmed kallas Stackens medlemmar till Stackens vårmöte 2015, torsdagen den 19:e mars klockan 19.30 på Kungliga Tekniska Högskolan, sal Q24.}

Var god betala din medlemsavgift om den inte redan är inbetald (eller du är hedersmedlem). Om du betalar den i tid (inkommen innan vårmötet) har du rätt att rösta på vårmötet. För studerande och andra med ont om pengar är avgiften 115 kronor, för övriga är den 215 kronor.

Medlemsavgiften betalas till PG 433 01 15-9, Datorföreningen Stacken. Glöm inte att ange namn och username - samtligas, om du betalar för flera personer. Eventuella frågor kan ställas till styrelsen@stacken.kth.se

~\\
Förslag till dagordning:

\begin{multicols}{2}
\begin{enumerate}
	\itemsep0em
	\item  Mötets öppnande
	\item  Val av mötesordförande
	\item  Val av mötessekreterare
	\item  Frågan om mötets stadgeenliga utlysande
	\item  Frågan om dagordningens godkännande
	\item  Val av justeringsmän
	\item  Tillkännagivande av röstlängd
	\item  Verksamhetsberättelse
	\item  Balansräkning
	\item  Revisionsberättelse
	\item  Ansvarsfrihet för 2014 års styrelse
	\item  Övriga frågor
	\item  Mötets avslutande
\end{enumerate}
\end{multicols}

\closing{}
\enlargethispage{3\baselineskip}

\end{letter}
\end{document}
